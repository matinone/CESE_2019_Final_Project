\chapter{Conclusiones}

En este capítulo se mencionan los aspectos más importantes del trabajo realizado y se contemplan los próximos pasos a seguir en el desarrollo.

\section{Conclusiones generales }

En este trabajo se logró implementar de manera exitosa un módulo capaz de dotar de conectividad WiFi y Bluetooth a un electrodoméstico. Para ello se diseñaron e implementaron diferentes módulos de firmware que permiten comunicarse con el usuario mediante la recepción de comandos por WiFi/Bluetooth, los cuales luego son retransmitidos al electrodoméstico (emulado por otro microcontrolador). También permiten que el fabricante cuente con información de estado en tiempo real de sus electrodomésticos, a través del envío y procesamiento de datos en Google Cloud Platform.

Durante el desarrollo del trabajo, fueron indispensables los conocimientos adquiridos en las diferentes materias de la Carrera de Especialización en Sistemas Embebidos, entre las que se destacan las siguientes:

\begin{itemize}
	\item Sistemas operativos de tiempo real I y II: constituyeron el primer contacto con un RTOS y permitieron aprender acerca del uso de tareas, recursos de sincronización y comunicación entre ellas, y uso de memoria en el contexto de un RTOS. 
	\item Protocolos de comunicación en sistemas embebidos: la interfaces de comunicación vistas (WiFi, Bluetooth, I2C) fueron una parte integral del trabajo realizado, al igual que otros protocolos de nivel superior (como HTTP para la capa de aplicación).
	\item Ingeniería de software en sistemas embebidos: permitió adquirir buenas prácticas de desarrollo de software, y elaborar el diseño de algunos módulos de firmware en una etapa temprana del trabajo.
\end{itemize}

\section{Próximos pasos}

El próximo paso lógico para el trabajo realizado consiste en la integración del módulo a un electrodoméstico real, para demostrar así el valor que aporta en un entorno más realista. Esto implicaría también el diseño y fabricación de un circuito impreso que se adapte al electrodoméstico seleccionado.

Además, existen una serie de funcionalidades adicionales que sería deseable implementar a futuro, a los fines de contar con un desarrollo aún más completo. Algunas de estas funcionalidades se presentan a continuación:

\begin{itemize}
	\item Agregar soporte para más interfaces de comunicación con el electrodoméstico: las pruebas se realizaron sólo con I2C, por lo que sería deseable agregar al menos SPI y UART, e incluso evaluar otros protocolos adicionales.
	\item Mejorar servidor web: algunas funcionalidades que se le podrían incorporar son la capacidad de mostrar las redes WiFi disponibles y el uso de encriptación para proteger la información que se intercambia al configurar las credenciales.
	\item Realizar actualizaciones remotas: conocidas como actualizaciones OTA (\emph{over the air}), permitirían que el fabricante actualice el firmware de todos sus electrodomésticos conectados de manera remota. El \emph{framework} del microcontrolador utilizado tiene soporte para recibir estas actualizaciones, y Google Cloud permite enviarlas, por lo que sería factible implementarlo.  
\end{itemize}











